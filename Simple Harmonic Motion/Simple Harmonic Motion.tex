\documentclass{article}

\title{Estimating the Spring Constant \\ through Simple Harmonic Motion \\ PHYS 101}

\author{Harlan Haskins, Jai Mehra, Matthew Vogt, Joshua Singh}

\date{\today}
\usepackage{pgfplots, pgfplotstable}
\usepackage{amsmath}

\begin{document}

\maketitle

\begin{center}
\begin{tabular}{l r}
Date Performed: & April 22nd, 2015 \\
Instructor: & Professor Andersen \\
Table: & 2
\end{tabular}
\end{center}

\pagebreak

\section{Objective}

To calculate $k$, the spring constant of a spring, using two methods:
\begin{enumerate}
	\item \emph{Static Method}: Measure the extension of the spring when various masses are added to find the spring constant.
	\item \emph{Dynamic Method}: Measure the period of simple harmonic motion for different added masses, and use a graphical technique to find the spring constant.
\end{enumerate}
Instead of plugging the values we get directly into equations, we use a graphical technique which will give us a visual representation of the uncertainty and the linear graph.
\begin{center}\end{center}

\section{Materials}
\begin{itemize}
	\item 2 $1m$ Metal Bars
	\item 1 Table and Bar Clamp
	\item 1 $90^{\circ}$ Bar Clamp
	\item 1 Spring
	\item 1 Weight Holder (weighs $5g$ itself)
	\item Various weights ranging from $10g$ – $50g$
	\item 1 Stopwatch
\end{itemize}

\section{Procedure}
Attach the table and bar clamp to the table and attach the bar. Attach the second bar to the first bar horizontally (to make a rod to which to attach the spring) using the 90 degree bar clamp. Place the spring on the horizontal rod and attach a mass on the bottom of the spring. For the \emph{Static Method}, measure the length of the spring in equilibrium with the meter stick. For the \emph{Dynamic Method}, measure the time period of some number of oscillations with the stopwatch. Write down the results and graph them using:

\begin{center}$T=2\pi\sqrt{\frac{m}{k}}$\end{center}

	which can be converted into
	
\begin{center}$T^{2}=\frac{(2\pi)^{2}}{k}m$\end{center}

where $T=\frac{oscillations}{time}$.

This equation can be graphed with $T$ on the y-axis and $m$ on the x-axis such that the graph would have a slope of $\frac{2\pi}{k}$.

\section{Experimental Data}
After recording the \emph{Static Method} and \emph{Dynamic Method} with
	many different masses, we came to this set of data, where $d_{n}$ represents the distance measured in the \emph{Static Method} for the $n$th trial, in $cm$, and $t_{n}$ represents the time measured in the \emph{Dynamic Method} for the $n$th trial in $s$.

\begin{center}
\begin{table}[h]
\begin{tabular}{|l|l|l|l|l|l|l|l|l|l|l|l|l|l|}
\hline
mass & $d_1$ & $d_2$ & $d_3$ & $d_4$ & $d_5$ & $d_{avg}$ & $t_1$ & $t_2$ & $t_3$ & $t_4$ & $t_5$ & $t_{avg}$ \\ \hline 
20g & 23.5 & 25 & 23.75 & 23.7 & 24.1 & 24.01 & 5.29 & 5.37 & 5.28 & 5.25 & 5.32 & 5.30 \\ \hline
50g & 27 & 26.5 & 26.7 & 27.2 & 26.9 & 26.86 & 6.32 & 6.15 & 6.29 & 6.25 & 6.18 & 6.24 \\ \hline
100g & 32.7 & 32.4 & 33.3 & 32.9 & 33.3 & 32.92 & 7.72 & 7.72 & 7.75 & 7.78 & 7.84 & 7.76 \\ \hline
150g & 39 & 39.1 & 39.3 & 39.4 & 39.4 & 39.24 & 8.71 & 8.94 & 9.00 & 8.93 & 9.03 & 8.92 \\ \hline
200g & 45.3 & 45.2 & 45.7 & 45.6 & 45.5 & 45.46 & 10.00 & 9.94 & 10.00 & 9.91 & 10.06 & 9.98 \\ \hline

\end{tabular}
\end{table}
\end{center}

\pgfplotstableread {
X Y sigma
20 28.1 0.432
50 38.9 0.802
100 60.2 0.699
150 79.6 1.991
200 99.6 1.045
} \oscillationtable

\begin{center}
\begin{tikzpicture}
\begin{axis} [
	width=1.0\textwidth,
    height=0.6\textwidth,
    title={Period of Oscillation vs Mass},
    xlabel={Mass [g]},
    ylabel={Oscillation Period $[\frac{osc^2}{s^2}]$},
    legend pos=north west,
    ymajorgrids=true,
    xmajorgrids=true,
    grid style=dashed,
]
 
\addplot [only marks, mark=square] table[y error=sigma] {\oscillationtable};
\addplot [thick, blue] table[
    y={create col/linear regression={y=Y}}
]{\oscillationtable};
\addlegendentry{$y(x)$}
\addlegendentry{
$\pgfmathprintnumber{\pgfplotstableregressiona}x
\pgfmathprintnumber[print sign]{\pgfplotstableregressionb}$}
\end{axis}
\end{tikzpicture}
\end{center}

We can then estimate $k$ with the slope of this graph.\footnote{This value clashes with the value found by mathematically solving for $k$, likely due to an error in the graph.}

\begin{center}
\begin{align*}
0.4 &= \frac{2\pi}{k} \\
k &= \frac{2\pi}{0.4} \\
k &= 15.7
\end{align*}
\end{center}

We can also estimate $k$ by looking at a graph of stretch distance vs mass.

\pgfplotstableread {
X Y
20 24.01
50 26.86
100 32.92
150 39.24
200 45.46
} \stretchtable

\begin{center}
\begin{tikzpicture}
\begin{axis} [
	width=1.0\textwidth,
    height=0.6\textwidth,
    title={Stretch Distance vs Mass},
    xlabel={Mass [g]},
    ylabel={Stretch Distance $[cm]$},
    legend pos=north west,
    ymajorgrids=true,
    xmajorgrids=true,
    grid style=dashed,
]
 
\addplot [only marks, mark=square] table[y error=sigma] {\stretchtable};
\addplot [thick, blue] table[
    y={create col/linear regression={y=Y}}
]{\stretchtable};
\addlegendentry{$y(x)$}
\addlegendentry{
$\pgfmathprintnumber{\pgfplotstableregressiona}x
\pgfmathprintnumber[print sign]{\pgfplotstableregressionb}$}
\end{axis}
\end{tikzpicture}
\end{center}

We can then estimate $k$ with the \emph{inverse} of the slope of this graph.

\begin{center}
\begin{align*}
k &= \frac{1}{0.12} \\
k &= 8.\overline{3}
\end{align*}
\end{center}

\pagebreak

\section{Results and Conclusions}

We ran the necessary calculations on $200g$ $(0.2kg)$ masses for our final answer, but the provided graphs scale linearly. Using the \emph{Static Method}:

\begin{align*}
F &= kx \\
(0.2m)(9.8\frac{m}{s^2}) &= k(0.45m-0.21m) \\
k &= \frac{(0.2m)(9.8\frac{m}{s^2})}{0.24m} \\
k &= 8.06\frac{N}{m}
\end{align*}

we estimated the spring to have a spring constant of $8.06\frac{N}{m}$.
Using the \emph{Dynamic Method} where $T$ is the period of oscillation:

\begin{align*}
T^{2} &= \frac{2\pi^{2}}{k}m \\
\frac{10}{10.0}\frac{osc}{s}^2 &= \frac{2\pi^{2}}{k}(0.2kg) \\
1.0 &= \frac{(2\pi)^{2}}{k}(0.2kg) \\
k &= 2\pi^2(0.2g) \\
k &= 7.895\frac{N}{m}
\end{align*}

the estimate becomes $7.895\frac{N}{m}$. The two forms of estimation agreed with each other to within $\pm{0.165}$.
\end{document}
